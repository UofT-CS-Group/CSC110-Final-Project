\documentclass[fontsize=11pt]{article}
\usepackage{amsmath}
\usepackage[utf8]{inputenc}
\usepackage[margin=0.75in]{geometry}
\usepackage{parskip}
\setlength\parindent{0pt}

\title{CSC110 Project Proposal\\ Educational Crisis - A Closer Examination on the Correlations Between Covid-19 and School Closures Around the Globe}
\author{Shouyi (Ray) Hung, Yuhan (Charlotte) Chen, Mengyuan (Alyssa) Li, and Fuyang (Scott) Cui}
\date{\today}

\begin{document}
\maketitle

\section*{Problem Description and Research Question}

As we move into the 21st century, educational qualities all around the world had experienced drastic changes as time passed. Under such circumstances, countless goals proposed by officials of different regions and countries were launched. An example of these goals would be Sustainable Development Goal Target 4, which is, Quality Education.

However, as we mark the end of a decade with the Year 2020, an unforeseen pandemic named COVID-19 disrupted the lives of millions of people all around the world. COVID-19 had influenced our lives in a plethora of ways. It not only posed severe health threats for humans in general, political and socio-economic impacts were also observed consistently in the months after the original discovery of COVID-19. One of the main socio-economic impacts that the COVID-19 pandemic posed on our society is the educational crisis.

\textbf{
    Our investigation aims to identify and examine the impacts that COVID-19 had brought to the educational sector of countries. To be more specific, we will focus on the relationships between the COVID-19 pandemic and school closures all around the world.
}

Our main motivation for investigating this topic is that as a group of international students, COVID-19 thoroughly changed many aspects of our life, and one of the most significant changes to us is school closures. We mostly studied in a remote environment, where physical contact is minimized. But later that year, some of these schools returned to the traditional face-to-face lectures. Therefore, we are curious about the correlation between COVID-19 and school closures, and the impacts of these changes on our mental health and education quality.

We hope that this project will provide some insights into the impacts on the educational scheme brought by this catastrophic event. Thereby, we may be able to predict how future pandemics affect our education and how to mitigate those potential negative influences.

\newpage

\section*{Dataset Description}

We have identified two main datasets that will be relevant for our project’s implementation.

These are:

\begin{enumerate}
    \item
        Global School Closures for COVID-19 – Obtained from Kaggle, compiled by Saleh Ahmed Rony, sourced from UNESCO

    \item
        COVID-19 Data Repository by the Center for Systems Science and Engineering (CSSE) at Johns Hopkins University – Obtained from GitHub, compiled by JHU, sourced from WHO, ECDC, DXY, US CDC, etc.
\end{enumerate}

Both datasets will be stored in a Comma Separated Value file, which will allow us to read from them easily through Python’s csv library.

Furthermore, both datasets are very credible as they are sourced from multiple sites, including but not limited to WHO, ECDC, and US CDC. Furthermore, these datasets are also licensed under the Creative Commons Attribution 4.0 International (CC BY 4.0), which allows us to utilize these data for our own needs.

The datasets that we have downloaded and utilized in this project were chosen because they were compiled in a way that allows easy access and manipulation. By using datasets that are already organized could improve the efficiency and robustness of our program.

\newpage


\section*{Computational Plan}

Our project will be separated into 2 parts: data processing and Graphical User Interface (GUI) implementation.

\subsection*{Processing Data}

To make those datasets that we have downloaded tidy and utilize them in our program, we will firstly read the data into Python with help of the Python csv library.

Then, we will convert raw data into many data classes so we could easily manipulate those data in practice.

\subsection*{GUI Implementation}

Currently, we plan to use \emph{PyQt5} to generate an interactive user interface, \emph{Matplotlib} to plot graphs, and some other helper tools like Qt Designer and other libraries.

We will implement the following functionalities, or parts:

\begin{itemize}
    \item
        Interactive COVID-19 cases and deaths and school closures visualizations supporting filtering, grouping, sorting functionalities, etc.

    \item
        Interactive table view of the visualizations above.
\end{itemize}

With a GUI, users of our application could quickly scan the data that we have provided and create their own plots by filtering the data by their will. Also, they could check which point on the graph may correspond to which observation.

With the help of our program, people could view, compare, and even predict the trend of the COVID-19 pandemic. Furthermore, we can also answer our research question by quickly filtering countries around the world and attempt to generalize a relationship between the trend of COVID-19 and school closures.

With these technical implementations, we believe that the impacts brought by the COVID-19 Pandemic would be conspicuous and people could formulate solutions towards the impacts illustrated.

\newpage

\section*{References}

“Coronavirus.” World Health Organization, WHO, 10 Jan. 2020, www.who.int/health-topics/coronavirus\#tab=tab\_1.

“CSV File Reading and Writing — Python 3.10.0 Documentation.” Python Documentation, docs.python.org/3/library/csv.html. Accessed 30 Oct. 2021.

Johns Hopkins University. “GitHub - CSSEGISandData/COVID-19: Novel Coronavirus (COVID-19) Cases, Provided by JHU CSSE.” GitHub, CSSEGISandData, github.com/CSSEGISandData/COVID-19. Accessed 30 Oct. 2021.

Kaggle | Global School Closures for COVID-19. Saleh Ahmed Rony, www.kaggle.com/salehahmedrony/global-school-closures-covid19. Accessed 30 Oct. 2021.

“Pygame v2.0.1.Dev1 Documentation.” Pygame, www.pygame.org/docs. Accessed 30 Oct. 2021.

“Scikit-Learn 1.0.1 Documentation.” Scikit-Learn, scikit-learn.org/stable/index.html. Accessed 30 Oct. 2021.

“Sustainable Development Goal 4 (SDG 4) | Education within the 2030 Agenda for Sustainable Development.” SDG 4 Education 2030, sdg4education2030.org/the-goal. Accessed 30 Oct. 2021.

\end{document}
